\documentclass[a4paper]{article}
%% Language and font encodings
\usepackage[english]{babel}
\usepackage[utf8x]{inputenc}
\usepackage[T1]{fontenc}

%% Sets page size and margins
\usepackage[letterpaper, portrait, margin=1in,top=1in,bottom=1.5in]{geometry}

%% Useful packages
\usepackage{amsmath}
\usepackage{amssymb}
\usepackage{amsthm}
\usepackage{amsfonts}
\usepackage{mathrsfs}
\usepackage{tikz}
\usepackage{graphicx}
\usepackage[shortlabels]{enumitem}
\newenvironment{exercise}[1]{\textbf{#1.}}

\begin{document}

\begin{flushright}
Cory Glover\\
Math 522\\
1/16/20
\end{flushright}


\begin{exercise}{2.9}
Let $G$ be an undirected graph. In this case, begin the breadth first search and view all the neighbors of a given node. We call this node $g^0$ and the set of its neighbors $g^0_n$. Then we check the each of the neighbors. As we travel to them, we label each one $g^1$ and the set of its neighbors $g^1_n$. If $g^0_n\cap g^1_n\neq\emptyset$, then there is a cycle. Once we have passed through all neighbors, we check the neighbors of the first node in $g^0_n$ and label this node $g^0_n$ and repeat the process. Further, if a node in is the neighborhood of the current node $g^i$ and is not the previous node $g^{i-1}$ (or $g^{i+1}$), then there is a cycle.
\end{exercise}

\begin{exercise}{3.1}
Let $X_1, X_2, X_3$ be random variables. We define the following properties:
\begin{align}
P(X_1=x_0)&=.5\\
P(X_1=x_1)&=.5\\
P(X_2=x_0)&=.7\\
P(X_2=x_1)&=.3\\
P(X_3=x_0)&=.2\\
P(X_3=x_1)&=.8.
\end{align}
As we can see, all the variables are independent. Then we define the probabilities
\begin{align}
P(X_1=x_0,X_2=x_0\mid X_3=x_0)&=.13\\
P(X_1=x_0,X_2=x_0\mid X_3=x_1)&=.02\\
P(X_1=x_1,X_2=x_1\mid X_3=x_0)&=.3\\
P(X_1=x_1,X_2=x_1\mid X_3=x_1)&=.05.
\end{align}
Thus we see that $X_1,X_2$ is dependent on $X_3$.
\end{exercise}

\begin{exercise}{3.2}
\begin{enumerate}
\item Let $X_1,...,X_n$ be random variables and $C$ be a class variable. Assume that $(X_i\perp\mathbf{X}_{-i}\mid C)$ for all $i$ where $\mathbf{X}_{-i}=\{X_1,...,X_n\}-\{X_i\}$. Then
\begin{align}
P(C,X_1,...,X_n)&=P(C)P(X_1\mid C)P(X_2,...,X_n\mid X_1,C)\\
&=P(C)P(X_1\mid C)P(X_2\mid X_1,C)P(X_3,...,X_n\mid X_1,X_2,C)\\
&=P(C)\prod_{i=1}^nP(X_i\mid C,X_{i-1},...,X_1)\\
&=P(C)\prod_{i=1}^nP(X_i\mid C),
\end{align}
where the last step used the assumption of independence.

\item Using the property proven in part (i), we see that
\begin{align}
\frac{P(C=c_1\mid X_1,...,X_n)}{P(C=c_2\mid X_1,...,X_n)}&=\frac{\frac{P(X_1,...,X_n\mid C=c_1)P(C=c_1)}{P(X_1,...,X_n)}}{\frac{P(X_1,...,X_n\mid C=c_2)P(C=c_2)}{P(X_1,...,X_n)}}\\
&=\frac{P(X_1,...,X_n\mid C=c_1)P(C=c_1)}{P(X_1,...,X_n\mid C=c_2)P(C=c_2)}\\
&=\frac{P(C=c_1)}{P(C=c_2)}\prod_{i=1}^n\frac{P(X_i\mid C=c_1)}{P(X_i\mid C=c_2)}.
\end{align}
\end{enumerate}
\end{exercise}

\begin{exercise}{3.3}
\begin{enumerate}
\item While having an earthquake does not eliminate the possibility of a burglary, having an earthquake causes the alarm to go off. So if an alarm goes off and there is an earthquake, it decreases the probability that a burglary set off the alarm because the earthquake did happen and was able to set the alarm off.

\item Assume that $P(a^1\mid b^1,e^1)=P(a^1\mid b^0,e^1)=1$. Then
\begin{align}
P(b^1\mid a^1,e^1)&=\frac{P(b^1)P(a^1,e^1\mid b^1)}{P(b^0)P(a^1,e^1\mid b^0)+P(b^1)P(a^1,e^1\mid b^1)}\\&=\frac{P(b^1)P(a^1\mid e^1,b^1)}{P(b^0)P(a^1\mid e^1,b^0)+P(b^1)P(a^1\mid e^1,b^1)}\\&=\frac{P(b^1)}{P(b^0)+P(b^1)}=P(b^1).\end{align}
\end{enumerate}
\end{exercise}

\begin{exercise}{3.}

\end{exercise}

\begin{exercise}{3.}

\end{exercise}

\end{document}