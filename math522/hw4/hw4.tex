 \documentclass[a4paper]{article}
%% Language and font encodings
\usepackage[english]{babel}
\usepackage[utf8x]{inputenc}
\usepackage[T1]{fontenc}

%% Sets page size and margins
\usepackage[letterpaper, portrait, margin=1in,top=1in,bottom=1.5in]{geometry}

%% Useful packages
\usepackage{amsmath}
\usepackage{amssymb}
\usepackage{amsthm}
\usepackage{amsfonts}
\usepackage{mathrsfs}
\usepackage{tikz}
\usepackage{graphicx}
\usepackage[shortlabels]{enumitem}
\newenvironment{exercise}[1]{\textbf{#1.}}

\begin{document}

\begin{flushright}
Cory Glover\\
1/24/20\\
Math 522
\end{flushright}

\begin{center}
HW 4
\end{center}

\begin{exercise}{3.25}
Let $\mathscr{K}$ be the graph returned by the PDAG algorithm. We work by way of contradiction. Assume there exists a minimal loop $X_1-X_2-\dotsb-X_k-X_1$ of length greater than or equal to 4. Recall that the graph $\mathscr{K}$ is acyclic. Thus, there must exist at least one directed edge in the loop (or else it would be a cycle). We then work by cases:
\begin{enumerate}
\item Assume there is an immorality in the path. Thus there exists $X_1-\dotsb-X_{i-1}\rightarrow X_i\leftarrow X_{i+1}-\dotsb-X_k-X_1$. 
\end{enumerate}
\end{exercise}

\end{document}