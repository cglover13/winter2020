 \documentclass[a4paper]{article}
%% Language and font encodings
\usepackage[english]{babel}
\usepackage[utf8x]{inputenc}
\usepackage[T1]{fontenc}

%% Sets page size and margins
\usepackage[letterpaper, portrait, margin=1in,top=1in,bottom=1.5in]{geometry}

%% Useful packages
\usepackage{amsmath}
\usepackage{amssymb}
\usepackage{amsthm}
\usepackage{amsfonts}
\usepackage{mathrsfs}
\usepackage{tikz}
\usepackage{graphicx}
\usepackage[shortlabels]{enumitem}
\newenvironment{exercise}[1]{\textbf{#1.}}

\begin{document}

\begin{flushright}
Cory Glover\\
2/4/20\\
Math 525
\end{flushright}

\begin{center}
HW 11
\end{center}

\begin{exercise}{7.15}
Recall that $\mathbf{\sigma}=(\mathbf{D}-\alpha\mathbf{A})^{-1}$. 
This means that
\begin{align}
\sigma&=(\mathbf{D}-\alpha\mathbf{A})^{-1}\\
\sigma^{-1}&=\mathbf{D}-\alpha\mathbf{A}\\
\sigma^{-1}\mathbf{D}^{-1}&=\mathbf{I}-\alpha\mathbf{A}\mathbf{D}^{-1}\\
(\mathbf{D}\sigma)^{-1}&=\mathbf{I}-\alpha\mathbf{A}\mathbf{D}^{-1}.
\end{align}
Recall that the pagerank centrality is given by $\mathbf{x}=(\mathbf{I}-\alpha\mathbf{A}\mathbf{D}^{-1})^{-1}\mathbf{1}$. This means that
\begin{align}
\mathbf{x}&=(\mathbf{I}-\alpha\mathbf{A}\mathbf{D}^{-1})^{-1}\mathbf{1}\\
&=\mathbf{D}\sigma\mathbf{1}\\
\mathbf{D}^{-1}\mathbf{x}&=\sigma\mathbf{1}.
\end{align}
Then we see that $\sigma_i=\sum_j\sigma_{ij}$ is just the $i^{th}$ entry of $\sigma\mathbf{1}$ since we are summing over the columns. Note that since $D$ is diagonal, the $D^{-1}$ is diagonal with diagonal entries $\frac{1}{k_i}$. So the $i^{th}$ entry of $\mathbf{D}^{-1}\mathbf{x}$ is $\frac{1}{k_i}x_i$. So $\sigma_i=\frac{1}{k_i}x_i$. Hence $\sigma_i$ is just the pagerank centrality of $i$ divided by the degree of node $i$.
\end{exercise}

\begin{exercise}{7.16}
\begin{enumerate}
\item 
\begin{tabular}{c|c|c}
Group&$e_r$&$a_r$\\
\hline
Black&.129&.177\\
Hispanic&.0785&.147\\
White&.153&.247\\
Other&.008&.0605
\end{tabular}
So $Q=.251$.

\item 
\begin{tabular}{c|c|c}
Group&$e_r$&$a_r$\\
\hline
Democrat&.125&.22\\
Independent&.03&.175\\
Republican&.03&.245
\end{tabular}
So $Q=.046$.

\item Based on the modularity values, it is clear that people associate themselves with people similar to themselves. This trend is strong in marriages than politics.

\end{enumerate}


\end{exercise}

\end{document}