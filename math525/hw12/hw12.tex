\documentclass[11pt]{article}
\usepackage[english]{babel}
\usepackage[utf8x]{inputenc}
\usepackage[T1]{fontenc}

%% Sets page size and margins
\usepackage[letterpaper, portrait, margin=1in,top=1in,bottom=1.5in]{geometry}

%% Useful packages
\usepackage{amsmath}
\usepackage{amssymb}
\usepackage{amsthm}
\usepackage{amsfonts}
\usepackage{mathrsfs}
\usepackage{tikz}
\usepackage{graphicx}
\usepackage[shortlabels]{enumitem}

\newenvironment{exercise}[1]{\textbf{#1.}}

\begin{document}

\begin{flushright}
Cory Glover\\
2/6/20\\
Math 525
\end{flushright}

\begin{center}
\textbf{HW 12}
\end{center}

\begin{exercise}{8.1}
\begin{enumerate}
\item $O(n)$
\item $O(n)$
\end{enumerate}
\end{exercise}

\begin{exercise}{8.3}
\begin{enumerate}
\item $O(n^2)$
\item $O(m)$
\item $O(m)$. Consider the vector multiplication of the modularity matrix. Since vector multiplication is distributive, we can multiply the vector by the adjacency matrix using list format, which as shown above is $O(m)$. Then to calculate $\frac{k_ik_j}{2m}$ for every combination of $i$ and $j$ takes $n$ calculations of $O(m/n)$, since we are finding how many nodes are connected to another node. So in total this takes $O(m)$. So we have 2 operations of complexity $O(m)$, thus the whole process is $O(m)$.
\end{enumerate}
\end{exercise}
\end{document}