 \documentclass[a4paper]{article}
%% Language and font encodings
\usepackage[english]{babel}
\usepackage[utf8x]{inputenc}
\usepackage[T1]{fontenc}

%% Sets page size and margins
\usepackage[letterpaper, portrait, margin=1in,top=1in,bottom=1.5in]{geometry}

%% Useful packages
\usepackage{amsmath}
\usepackage{amssymb}
\usepackage{amsthm}
\usepackage{amsfonts}
\usepackage{mathrsfs}
\usepackage{tikz}
\usepackage{graphicx}
\usepackage[shortlabels]{enumitem}
\newenvironment{exercise}[1]{\textbf{#1.}}

\begin{document}

\begin{flushright}
Cory Glover\\
Math 525\\
1/10/19
\end{flushright}

\begin{exercise}{1}
B
\end{exercise}

\begin{exercise}{2}
Consider a graph $G=(V,E)$ where $V=\{v_1,v_2,...,v_n\}$. Without loss of generality, select vertex $v_1$ and assume that there is an edge between $v_1$ and $v_2,...,v_b$. Make each of these edges directed towards $v_1$. Order the set of vertices pointing towards $v_1$ and select the first (without loss of generality we name this $v_2$). Make each undirected edge connected to $v_2$ a directed edge directed towards $v_2$. Repeat this process for each vertex in the set of vertices pointing towards $v_1$. 

Now select $v_2$ and consider all the vertices connect to $v_2$. Order the set of vertices pointing towards $v_2$ and select the first. Repeat the process describe for this vertex as the one used for the set of vertices connected to $v_2$. Repeat for all vertices connected to $v_2$. Then consider the vertices connected to $v_3$ and select the first, and repeat the process described. Repeat for all vertices connected to $v_1$.

Consider a graph $G=(V,E)$ where $V=\{v_1,v_2,...,v_n\}$. Without loss of generality, select vertex $v_1$ and assume that there is an edge between $v_1$ and $v_2,...,v_b$. Make each of these edges directed towards $v_1$. Now consider the set of all vertices which have at least one directed edge and order them. Order these vertices and select the first. Make every undirected edge connected to this vertex a directed edge pointed towards the vertex in question. Repeat for all vertices in the order set. Once all vertices have been considered, consider the new set of all vertices which have at least one directed edge and order them. Repeat the process on this set until all vertices in the set have been considered. Continue creating this set and directing edges until all edges have been directed. In the case that the graph has disconnected components, perform this process on each component.

We now show the graph created by the algorithm is acyclic. Drawing this graph, we order the nodes $v_1$ to $v_n$ from bottom to top. Then all edges will point down (since the algorithm added edges in the order $v_1$ to $v_n$. Then by page 112, this graph is acyclic.
\end{exercise}

\begin{exercise}{3}
\begin{enumerate}
\item Undirected, cyclic
\item Directed, cyclic
\item Undirected, acyclic, planar
\item Undirected, acyclic, planar
\item Undirected, cyclic
\item Citation network
\item Airports connected by the airplanes that fly between them
\item Bones in the arm (joints are nodes)
\item Constellation
\item Recipes are one type of node and foods are another. They are connected if the food is used in the recipe
\end{enumerate}
\end{exercise}

\begin{exercise}{4}
\begin{enumerate}
\item $\begin{pmatrix}
0&1&0&0&1\\
0&0&1&0&0\\
1&0&0&0&1\\
0&1&1&0&0\\
0&0&0&0&0
\end{pmatrix}$

\item $\begin{pmatrix}
1&0&1&0&0\\
0&1&1&0&0\\
0&0&0&1&0\\
0&1&1&1&1
\end{pmatrix}
$
\item $\begin{pmatrix}
1&0&1&0&0\\
0&2&2&1&1\\
1&2&3&1&1\\
0&1&1&2&1\\
0&1&1&1&1
\end{pmatrix}$
\end{enumerate}
\end{exercise}


\end{document}