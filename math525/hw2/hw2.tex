 \documentclass[a4paper]{article}
%% Language and font encodings
\usepackage[english]{babel}
\usepackage[utf8x]{inputenc}
\usepackage[T1]{fontenc}

%% Sets page size and margins
\usepackage[letterpaper, portrait, margin=1in,top=1in,bottom=1.5in]{geometry}

%% Useful packages
\usepackage{amsmath}
\usepackage{amssymb}
\usepackage{amsthm}
\usepackage{amsfonts}
\usepackage{mathrsfs}
\usepackage{tikz}
\usepackage{graphicx}
\usepackage[shortlabels]{enumitem}
\newenvironment{exercise}[1]{\textbf{#1.}}

\begin{document}

\begin{flushright}
Cory Glover\\
Math 525\\
1/10/19
\end{flushright}

\begin{exercise}{1}
B
\end{exercise}

\begin{exercise}{2}
Consider a graph $G=(V,E)$ where $V=\{v_1,v_2,...,v_n\}$. Without loss of generality, select vertex $v_1$ and assume that there is an edge between $v_1$ and $v_2,...,v_b$. Make each of these edges directed towards $v_1$. Order the set of vertices pointing towards $v_1$ and select the first (without loss of generality we name this $v_2$). Make each undirected edge connected to $v_2$ a directed edge directed towards $v_2$. Repeat this process for each vertex in the set of vertices pointing towards $v_1$. 

Now select $v_2$ and consider all the vertices connect to $v_2$. Order the set of vertices pointing towards $v_2$ and select the first. Repeat the process describe for this vertex as the one used for the set of vertices connected to $v_2$. Repeat for all vertices connected to $v_2$. Then consider the vertices connected to $v_3$ and select the first, and repeat the process described. Repeat for all vertices connected to $v_1$.

Consider a graph $G=(V,E)$ where $V=\{v_1,v_2,...,v_n\}$. Without loss of generality, select vertex $v_1$ and assume that there is an edge between $v_1$ and $v_2,...,v_b$. Make each of these edges directed towards $v_1$. Now consider the set of all vertices which have at least one directed edge and order them. Order these vertices and select the first. Make every undirected edge connected to this vertex a directed edge pointed towards the vertex in question. Repeat for all vertices in the order set. Once all vertices have been considered, consider the new set of all vertices which have at least one directed edge and order them. Repeat the process on this set until all vertices in the set have been considered. Continue creating this set and directing edges until all edges have been directed. In the case that the graph has disconnected components, perform this process on each component.

We now show the graph created by the algorithm is acyclic. Drawing this graph, we order the nodes $v_1$ to $v_n$ from bottom to top. Then all edges will point down (since the algorithm added edges in the order $v_1$ to $v_n$. Then by page 112, this graph is acyclic.
\end{exercise}

\begin{exercise}{3}
Consider a simple network consisting of $n$ nodes in a simple component. The maximum number of edges it could have is $\sum_{i=1}^{n-1}i$. This is when every node will connect to every other node. So the first node will have $n-1$ edges coming out of it. For the second node, the node connected the first and second is already drawn, so there are $n-2$ new edges coming out of it. Continuing this procees to the $n^{th}$ node, we see that every node is already connected to the $n^{th}$ node. We add these to get the number of edges. 

The minimum number of edges it could have is $n-1$. Since there is a single component, we know that all the nodes must be connected to some other node in the component. We take the first node and connect it to the second, the second to the third, up to the $n^{th}$. Since no new edge is drawn out of the $n^{th}$ node, there are only $n-1$ edges.
\end{exercise}

\begin{exercise}{4}
\begin{enumerate}
\item $\begin{pmatrix}
0&1&0&0&1\\
0&0&1&0&0\\
1&0&0&0&1\\
0&1&1&0&0\\
0&0&0&0&0
\end{pmatrix}$

\item $\begin{pmatrix}
1&0&1&0&0\\
0&1&1&0&0\\
0&0&0&1&0\\
0&1&1&1&1
\end{pmatrix}
$
\item $\begin{pmatrix}
1&0&1&0&0\\
0&2&2&1&1\\
1&2&3&1&1\\
0&1&1&2&1\\
0&1&1&1&1
\end{pmatrix}$
\end{enumerate}
\end{exercise}


\end{document}