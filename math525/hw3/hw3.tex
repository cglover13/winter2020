 \documentclass[a4paper]{article}
%% Language and font encodings
\usepackage[english]{babel}
\usepackage[utf8x]{inputenc}
\usepackage[T1]{fontenc}

%% Sets page size and margins
\usepackage[letterpaper, portrait, margin=1in,top=1in,bottom=1.5in]{geometry}

%% Useful packages
\usepackage{amsmath}
\usepackage{amssymb}
\usepackage{amsthm}
\usepackage{amsfonts}
\usepackage{mathrsfs}
\usepackage{tikz}
\usepackage{graphicx}
\usepackage[shortlabels]{enumitem}
\newenvironment{exercise}[1]{\textbf{#1.}}

\begin{document}

\begin{flushright}
Cory Glover\\
Math 525\\
1/14/19
\end{flushright}

\begin{center}
HW 3
\end{center}

\begin{exercise}{6.4}
\begin{enumerate}
\item $A\mathbf{1}=\mathbf{k}$

\item $\frac{\mathbf{1}^TA\mathbf{1}}{2}$

\item $A^TA=N$

\item $\frac{tr(A^3)}{6}$

\end{enumerate}
\end{exercise}

\begin{exercise}{6.9}

\end{exercise}

\begin{exercise}{6.10}
If someone has an odd number of siblings, that means that all of their siblings also have an odd number of siblings. Thus, there is an even number of people with an odd number of siblings (since there are an even number of children). This will be true for all sets of siblings and the sum of even numbers is even. Thus the number of people with an odd number of siblings is even.
\end{exercise}
\end{document}