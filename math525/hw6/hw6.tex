 \documentclass[a4paper]{article}
%% Language and font encodings
\usepackage[english]{babel}
\usepackage[utf8x]{inputenc}
\usepackage[T1]{fontenc}

%% Sets page size and margins
\usepackage[letterpaper, portrait, margin=1in,top=1in,bottom=1.5in]{geometry}

%% Useful packages
\usepackage{amsmath}
\usepackage{amssymb}
\usepackage{amsthm}
\usepackage{amsfonts}
\usepackage{mathrsfs}
\usepackage{tikz}
\usepackage{graphicx}
\usepackage[shortlabels]{enumitem}
\newenvironment{exercise}[1]{\textbf{#1.}}

\begin{document}

\begin{flushright}
Cory Glover\\
Math 525\\
1/23/19
\end{flushright}

\begin{center}
HW 6
\end{center}

\begin{exercise}{1}
A
\end{exercise}

\begin{exercise}{2}

\end{exercise}

\begin{exercise}{3}
A network of basketball teams indicating wins as in arrows and loses as out arrows. Importance would be beating teams that are very good. For example, if you are a small school that wins a lot, but loses everytime to a very good school that plays more competitively, you are not as important. So eigenvector centrality is a better measure.
\end{exercise}

\begin{exercise}{4}
\begin{enumerate}
\item $c(1)=1/13, c(2)=3/13, c(3)=3/13, c(4)=2/13, c(5)=2/13, c(6)=2/13$
\item $c(1)=.455, c(2)=.233, c(3)=.119, c(4)=.537, c(5)=.4343, c(6)=.4971$
\item $\alpha=.1: c(1)=.149, c(2)=.181, c(3)=.179, c(4)=.163, c(5)=.164, c(6)=.164$.

$\alpha=.3: c(1)=.128, c(2)=.210, c(3)=.196, c(4)=.151, c(5)=.158, c(6)=.158$.

$\alpha=.5: c(1)=.124, c(2)=.241, c(3)=.207, c(4)=.140, c(5)=.149, c(6)=.149$.

As $\alpha$ increases, the importance of the most connectted nodes increases more and the less connected nodes goes down more.
\end{enumerate}
\end{exercise}

\end{document}