 \documentclass[a4paper]{article}
%% Language and font encodings
\usepackage[english]{babel}
\usepackage[utf8x]{inputenc}
\usepackage[T1]{fontenc}

%% Sets page size and margins
\usepackage[letterpaper, portrait, margin=1in,top=1in,bottom=1.5in]{geometry}

%% Useful packages
\usepackage{amsmath}
\usepackage{amssymb}
\usepackage{amsthm}
\usepackage{amsfonts}
\usepackage{mathrsfs}
\usepackage{tikz}
\usepackage{graphicx}
\usepackage[shortlabels]{enumitem}
\newenvironment{exercise}[1]{\textbf{#1.}}

\begin{document}

\begin{flushright}
Cory Glover\\
1/27/20\\
Math 540
\end{flushright}

\begin{center}
HW 10\\
2.6.2,3,14
\end{center}

\begin{exercise}{2.6.2}
For all the operations, the range and domain are both over the reals. It suffices to show that the second part of the definition is true for all operators.
\begin{enumerate}
\item Consider the operator $T(x_1,x_2)=(x_1)$. Then for some $\alpha,\beta\in\mathbb{R}$,
\begin{align}
T(\alpha(x_1,x_2)+\beta(y_1,y_2))&=T((\alpha x_1,\alpha x_2)+(\beta y_1,\beta y_2))\\&=T(\alpha x_1 + \beta y_1,\alpha x_2+\beta y_2)\\&=(\alpha x_1+\beta y_1,0)\\&=(\alpha x_1,0)+(\beta y_1,0)\\&=\alpha(x_1,0)+\beta(y_1,0)\\&=\alpha T(x_1,x_2)+\beta T(y_1,y_2).\end{align}
So $T$ is linear.

\item Consider the operator $T(x_1,x_2)=(0,x_2)$. Then for some $\alpha,\beta\in\mathbb{R}$,
\begin{align}
T(\alpha(x_1,x_2)+\beta(y_1,y_2))&=T((\alpha x_1,\alpha x_2)+(\beta y_1,\beta y_2))\\&=T(\alpha x_1 + \beta y_1,\alpha x_2+\beta y_2)\\&=(0,\alpha x_2+\beta y_2)\\&=(0,\alpha x_2)+(0,\beta y_2)\\&=\alpha(0,x_2)+\beta(0,y_2)\\&=\alpha T(x_1,x_2)+\beta T(y_1,y_2).\end{align}
So $T$ is linear.

\item Consider the operator $T(x_1,x_2)=(x_2,x_1)$. Then for some $\alpha,\beta\in\mathbb{R}$,
\begin{align}
T(\alpha(x_1,x_2)+\beta(y_1,y_2))&=T((\alpha x_1,\alpha x_2)+(\beta y_1,\beta y_2))\\
&=T(\alpha x_1+\beta y_1,\alpha x_2+\beta y_2)\\
&=(\alpha x_2+\beta y_2,\alpha x_1+\beta y_1)\\
&=(\alpha x_2,\alpha x_1)+(\beta y_2,\beta y_1)\\
&=\alpha (x_2,x_1)+\beta (y_2,y_1)\\
&=\alpha T(x_1,x_2)+\beta T(y_1,y_2).
\end{align}
So $T$ is linear.

\item Consider the operator $T(x_1,x_2)=(\gamma x_1,\gamma x_2)$. Then for some $\alpha,\beta\in\mathbb{R}$,
\begin{align}
T(\alpha(x_1,x_2)+\beta(y_1,y_2))&=T((\alpha x_1,\alpha x_2)+(\beta y_1,\beta y_2))\\
&=T(\alpha x_1+\beta y_1,\alpha x_2+\beta y_2)\\
&=(\gamma(\alpha x_1+\beta y_1),\gamma(\alpha x_2+\beta y_2))\\
&=(\gamma\alpha x_1+\gamma\beta y_1,\gamma\alpha x_2+\gamma\beta y_2)\\
&=(\gamma\alpha x_1,\gamma\alpha x_2)+(\gamma\beta y_1,\gamma\beta y_2)\\
&=\alpha(\gamma x_1,\gamma x_2)+\beta(\gamma y_1,\gamma y_2)\\
&=\alpha T(x_1,x_2)+\beta T(y_1,y_2).
\end{align}
So $T$ is linear.
\end{enumerate}
\end{exercise}

\begin{exercise}{2.6.3}
\begin{enumerate}
\item $D(T_1)=\mathbb{R}^2$, $R(T_1)=\{(x,0)\colon x\in\mathbb{R}\}$, $N(T_1)=\{(0,x)\colon x\in\mathbb{R}\}$.
\item $D(T_2)=\mathbb{R}^2$, $R(T_2)=\{(0,x)\colon x\in\mathbb{R}\}, N(T_2)=\{(x,0)\colon x\in\mathbb{R}\}$.
\item $D(T_3)=R(T_3)=\mathbb{R}^2$, $N(T_3)=\{(0,0)\}$.
\end{enumerate}
\end{exercise}

\begin{exercise}{2.6.14}
Let $T\colon X\rightarrow Y$ be a linear operator and $dim(X)=dim(Y)=n<\infty$. Assume that $R(T)=Y$. Thus, $T$ is onto. So for every $y\in Y$, there exists $x\in X$ such that $T(x)=y$. We now show that $T$ is injective. Assume that $y=y^*$. Since $Y$ has dimension $n$, we know there exists a basis such that
\[y=\alpha_1y_1+\dotsb+\alpha_nx_n.\]
We also know that $y^*$ has the same decomposition. So 
\begin{align}
y&=y^*\\
\alpha_1y_1+\dotsb+\alpha_ny_n&=\alpha_1y_1+\dotsb+\alpha_ny_n.
\end{align}
We then know each $y_1$ can be written as $T(x_1)$. So
\begin{align}
\alpha_1T(x_1)+\dotsb\alpha_nT(x_n)&=\alpha_1T(x_1)+\dotsb\alpha_nT(x_n)\\
T(\alpha_1x_1+\dotsb+\alpha_nx_n)&=T(\alpha_1x_1+\dotsb+\alpha_nx_n).
\end{align}
Since the decompostion of the domain values mapping to $y$ and $y^*$ are the same, we know that $x=x^*$ where $T(x)=y$ and $T(x^*)=y^*$. So $T$ is injective. Thus, $T$ is bijective and an inverse exists.

Now assume that $T^{-1}$ exists. Then $dim(D(T^{-1})=n<\infty$. Clearly $T$ exists and $(T^{-1})^{-1}=T$. Then by Theorem 2.6-10, we know that $dim(D(T^{-1}))=n$. Since $T^{-1}$ takes on values from $Y$, and $dim(Y)=n$, then $D(T^{-1})=Y$. Note that $R(T)=D(T^{-1})$. So $R(T)=Y$.
\end{exercise}

\end{document}