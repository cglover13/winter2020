 \documentclass[a4paper]{article}
%% Language and font encodings
\usepackage[english]{babel}
\usepackage[utf8x]{inputenc}
\usepackage[T1]{fontenc}

%% Sets page size and margins
\usepackage[letterpaper, portrait, margin=1in,top=1in,bottom=1.5in]{geometry}

%% Useful packages
\usepackage{amsmath}
\usepackage{amssymb}
\usepackage{amsthm}
\usepackage{amsfonts}
\usepackage{mathrsfs}
\usepackage{tikz}
\usepackage{graphicx}
\usepackage[shortlabels]{enumitem}
\newenvironment{exercise}[1]{\textbf{#1.}}

\begin{document}

\begin{flushright}
Cory Glover\\
1/27/20\\
Math 540
\end{flushright}

\begin{center}
HW 10\\
2.6.2,3,14
\end{center}

\begin{exercise}{2.6.2}
For all the operations, the range and domain are both over the reals. It suffices to show that the second part of the definition is true for all operators.
\begin{enumerate}
\item Consider the operator $T(x_1,x_2)=(x_1)$. Then for some $\alpha,\beta\in\mathbb{R}$,
\begin{align}
T(\alpha(x_1,x_2)+\beta(y_1,y_2))&=T((\alpha x_1,\alpha x_2)+(\beta y_1,\beta y_2))\\&=T(\alpha x_1 + \beta y_1,\alpha x_2+\beta y_2)\\&=(\alpha x_1+\beta y_1,0)\\&=(\alpha x_1,0)+(\beta y_1,0)\\&=\alpha(x_1,0)+\beta(y_1,0)\\&=\alpha T(x_1,x_2)+\beta T(y_1,y_2).\end{align}
\end{enumerate}
\end{exercise}

\end{document}