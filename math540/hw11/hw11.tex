 \documentclass[a4paper]{article}
%% Language and font encodings
\usepackage[english]{babel}
\usepackage[utf8x]{inputenc}
\usepackage[T1]{fontenc}

%% Sets page size and margins
\usepackage[letterpaper, portrait, margin=1in,top=1in,bottom=1.5in]{geometry}

%% Useful packages
\usepackage{amsmath}
\usepackage{amssymb}
\usepackage{amsthm}
\usepackage{amsfonts}
\usepackage{mathrsfs}
\usepackage{tikz}
\usepackage{graphicx}
\usepackage[shortlabels]{enumitem}
\newenvironment{exercise}[1]{\textbf{#1.}}

\begin{document}

\begin{flushright}
Cory Glover\\
1/27/20\\
Math 540
\end{flushright}

\begin{center}
HW 11\\
2.7.2,6,10,14\\
2.8.2,4,11
\end{center}

\begin{exercise}{2.7.2}
Let $X$ and $Y$ be normed spaces. Assume that a linear operator $T\colon X\rightarrow Y$ is bounded. Let $A\subseteq X$ be a set that is bounded. So for all $a\in A$, $\|a\|\leq M$ for some $M\in\mathbb{R}$ (or $\mathbb{C}$. The same proof follow either way). Let $b\in B$ such that $T(A)=B$. Thus, there exists an $a\in A$ such that $T(a)=b$. We know that $\|T(a)\|\leq c\|a\|$ for some $c$. Thus, $\|b\|\leq c\|a\|$. Since $A$ is bounded, then $\|b\|\leq c\|a\|\leq cM$. Since $cM\in\mathbb{R}$ and is constant, then $\|b\|$ is bounded. Since $b$ was arbitrary, $B$ must be bounded. So $T$ maps bounded sets into bounded sets.

Now assume that $T\colon X\rightarrow Y$ maps bounded sets in $X$ to bounded sets in $Y$. Let $x\in X$. Then $\|Tx\|=\|y\|$ where $y\in Y$. If $x$ is in some bounded set of $X$, then $\|Tx\|\leq c$ for some $c$ since $y$ must be in a bounded set. Let $d=\frac{c}{\|x\|}$. Then $\|Tx\|\leq d\|x\|$.

Now assme that $x$ is not in some bounded set of $X$. 
\end{exercise}

\begin{exercise}{2.7.6}
Let $T\colon l^\infty\rightarrow l^\infty$ defined on $y=(y_i)=Tx$, $y_i=\frac{x_i}{i}$ and $x=(x_i)$. We first show that $T$ is linear and bounded.
\begin{enumerate}
\item (Linear): Let $a,b\in l^\infty$ be denoted $a=(a_i),b=(b_i).$  Let $\alpha,\beta\in\mathbb{R}$. Then 
\begin{align}
T(\alpha a+\beta b)=T((\alpha a_i)+(\beta b_i))\\
&=T((\alpha a_i+\beta b_i))\\
&=(\frac{\alpha a_i+\beta b_i}{i})\\
&=(\frac{\alpha a_i}{i})+(\frac{\beta b_i}{i})\\
&=\alpha(\frac{a_i}{i})+\beta(\frac{b_i}{i})\\
&=\alpha T(a)+\beta T(b).
\end{align}
So $T$ is linear.

\item (Bounded): Note that $\|Tx\|=\|(\frac{x_i}{i})\|$ Since $(\frac{x_i}{i})\in l^\infty$, each $\frac{x_i}{i}$ is bounded by some $M$. Thus, $\|(\frac{x_i}{i})\|=\sup_i\|\frac{x_i}{i}\|\leq M$. Let $c=\frac{M}{\|x\|}$. Then $\|Tx\|=\|(\frac{x_i}{i})\|=\sup_i|\frac{x_i}{i}|\leq M=c\|x\|$. So $\|Tx\|\leq c\|x\|$. So $T$ is bounded.
\end{enumerate}

Thus we know that $T$ is a bounded linear operator. We now consider $R(T)$. We know every sequence in $R(T)$ is of the form $(\frac{x_i}{i})$. Consider the sequence of sequences $(x_i)$ where $(x_i)=( .)$ where the first $i$ terms are $\frac{1}{n}$. So $T(x_i)=(\frac{1}{n},\frac{1}{2n}
\end{exercise}

\end{document}