 \documentclass[a4paper]{article}
%% Language and font encodings
\usepackage[english]{babel}
\usepackage[utf8x]{inputenc}
\usepackage[T1]{fontenc}

%% Sets page size and margins
\usepackage[letterpaper, portrait, margin=1in,top=1in,bottom=1.5in]{geometry}

%% Useful packages
\usepackage{amsmath}
\usepackage{amssymb}
\usepackage{amsthm}
\usepackage{amsfonts}
\usepackage{mathrsfs}
\usepackage{tikz}
\usepackage{graphicx}
\usepackage[shortlabels]{enumitem}
\newenvironment{exercise}[1]{\textbf{#1.}}

\begin{document}

\begin{flushright}
Cory Glover\\
1/27/20\\
Math 540
\end{flushright}

\begin{center}
HW 11\\
2.7.2,6,10,14\\
2.8.2,4,11
\end{center}

\begin{exercise}{2.7.2}
Let $X$ and $Y$ be normed spaces. Assume that a linear operator $T\colon X\rightarrow Y$ is bounded. Let $A\subseteq X$ be a set that is bounded. So for all $a\in A$, $\|a\|\leq M$ for some $M\in\mathbb{R}$ (or $\mathbb{C}$. The same proof follow either way). Let $b\in B$ such that $T(A)=B$. Thus, there exists an $a\in A$ such that $T(a)=b$. We know that $\|T(a)\|\leq c\|a\|$ for some $c$. Thus, $\|b\|\leq c\|a\|$. Since $A$ is bounded, then $\|b\|\leq c\|a\|\leq cM$. Since $cM\in\mathbb{R}$ and is constant, then $\|b\|$ is bounded. Since $b$ was arbitrary, $B$ must be bounded. So $T$ maps bounded sets into bounded sets.

Now assume that $T\colon X\rightarrow Y$ maps bounded sets in $X$ to bounded sets in $Y$. Let $x\in X$. Then $\|Tx\|=\|y\|$ where $y\in Y$. If $x$ is in some bounded set of $X$, then $\|Tx\|\leq c$ for some $c$ since $y$ must be in a bounded set. Let $d=\frac{c}{\|x\|}$. Then $\|Tx\|\leq d\|x\|$.

Now assume that $x$ is not in some bounded set of $X$. So $\{x\}$ is unbounded. This means that $\|x\|=\infty$. This is a contradiction. So all $x$ are in some bounded set. Thus by the first part of the proof, $T$ is a bounded operator.
\end{exercise}

\begin{exercise}{2.7.6}
Let $T\colon l^\infty\rightarrow l^\infty$ defined on $y=(y_i)=Tx$, $y_i=\frac{x_i}{i}$ and $x=(x_i)$. We first show that $T$ is linear and bounded.
\begin{enumerate}
\item (Linear): Let $a,b\in l^\infty$ be denoted $a=(a_i),b=(b_i).$  Let $\alpha,\beta\in\mathbb{R}$. Then 
\begin{align}
T(\alpha a+\beta b)=T((\alpha a_i)+(\beta b_i))\\
&=T((\alpha a_i+\beta b_i))\\
&=(\frac{\alpha a_i+\beta b_i}{i})\\
&=(\frac{\alpha a_i}{i})+(\frac{\beta b_i}{i})\\
&=\alpha(\frac{a_i}{i})+\beta(\frac{b_i}{i})\\
&=\alpha T(a)+\beta T(b).
\end{align}
So $T$ is linear.

\item (Bounded): Note that $\|Tx\|=\|(\frac{x_i}{i})\|$ Since $(\frac{x_i}{i})\in l^\infty$, each $\frac{x_i}{i}$ is bounded by some $M$. Thus, $\|(\frac{x_i}{i})\|=\sup_i\|\frac{x_i}{i}\|\leq M$. Let $c=\frac{M}{\|x\|}$. Then $\|Tx\|=\|(\frac{x_i}{i})\|=\sup_i|\frac{x_i}{i}|\leq M=c\|x\|$. So $\|Tx\|\leq c\|x\|$. So $T$ is bounded.
\end{enumerate}

Thus we know that $T$ is a bounded linear operator. We now consider $R(T)$. We know every sequence in $R(T)$ is of the form $(\frac{x_i}{i})$. Consider the sequence of sequences $(x_i)$ where $(x_n)=(1,2,3,...,n,0,0,...)$. Then $Tx_n=(1,1,...,1,0,...)$. Note that as $n\rightarrow\infty$, $Tx_n\rightarrow (1,1,...)$. Since $(1,1,1,...)$ is bounded by 1, $(1,1,1,...)\in l^\infty$. However $T^{-1}(1,1,1,...)=(1,2,3,....)\notin X$. So $(1,1,1,...)\notin R(T)$. Thus, $R(T)$ is not closed.
\end{exercise}

\begin{exercise}{2.7.10}
Let $X$ be $C[0,1]$ and define $S\colon= y(s)=s\int_0^1x(t)dt$ and $T\colon=y(s)=sx(s)$. First note that
\begin{align}
TSx(s)=T(s\int_0^1x(t)dt)=s^2\int_0^1x(t)dt\\
STx(s)=S(sx(s))=s\int_0^1tx(t)dt.
\end{align}
So $ST\neq TS$.

Then we see that
\begin{align}
\|S\|&=\sup_{x\in\mathscr{D},\|x\|=1}\|Sx\|\\
&=\sup_{x\in\mathscr{D},\|x\|=1}\|s\int_0^1x(t)dt\|\\
&=\max_{s\in[0,1]}|s\int_0^1x(t)dt|\\
&=|\int_0^1x(t)|dt\\
&=\|x\|.\\
\|T\|&=\sup_{x\in\mathscr{D}(T),\|x\|=1}\|Tx\|\\
&=\max_{s\in[0,1]}|sx(s)|.\\
\|ST\|&=\max_{s\in[0,1]}|s\int_0^1tx(t)dt|\\
&=|\int_0^1tx(t)dt|\\
&=\|tx(t)\|\\
\|TS\|&=\max_{s\in[0,1]}|s^2\int_0^1x(t)dt|\\
&=|\int_0^1x(t)dt|\\
&=\|x\|.
\end{align}
\end{exercise}

\begin{exercise}{2.7.14}
Let $\|x\|_1=\sum_{k=1}^n|x_k|$ and $\|y\|_2=\sum_{j=1}^r|y_j|$. Let $A$ be a $r\times n$ matrix $A=(\alpha_{jk})$ which defines a linear operator from the vector space $X$ of all ordered $n$-tuples of numbers into the vector space $Y$ of all ordered $r$-tuples of numbers. Then we see that
\begin{align}
\|Ax\|_2&=\sum_{i=1}^r|\sum_{j=1}^n\alpha_{ij}x_j|\\
&\leq \sum_{i=1}^r\sum_{j=1}^n|\alpha_{ij}x_j|\\
&\leq \max_k\sum_{i=1}^r\sum_{j=1}^n|\alpha_{ik}x_j|\\
&\leq\max_k\sum_{i=1}^r|\alpha_{ik}|\sum_{j=1}^n|x_j|\\
&=\|A\|\|x\|_1.
\end{align}
So $\|A\|$ is compatible.
\end{exercise}

\begin{exercise}{2.8.2}
Let $y_0\in C[a,b]$. Then we see that for $\alpha,\beta\in\mathbb{R}$ (or $\mathbb{C}$), that
\begin{align}
f_1(\alpha x+\beta y)&=\int_a^b(\alpha x+\beta y)(t)y_0(t)dt\\
&=\int_a^b((\alpha x)(t)+\beta y(t))y_0(t)dt\\
&=\int_a^b(\alpha x(t)+\beta y(t))y_0(t)dt\\
&=\int_a^b \alpha x(t)y_0(t)+\beta y(t)y_0(t)dt\\
&=\int_a^b\alpha x(t)y_0(t)dt+\int_a^b\beta y(t)y_0(t)dt\\
&=\alpha\int_a^b x(t)y_0(t)+\beta\int_a^b y(t)y_0(t)dt\\
&=\alpha f_1(x)+\beta f_2(y).
\end{align}
So $f_1$ is linear. We now note that the continuity of $y_0(t)$ on a closed square implies that $y_0$ is bounded, say $|y_0(t)|\leq c$. So, 
\begin{align}
\|f_1(x)\|&=|\int_a^bx(t)y_0(t)dt|\\
&\leq c|\int_a^bx(t)|\\
&\leq c(b-a)\max_{t\in[a,b]}|x(t)|\\
&=c(b-a)\|x\|.
\end{align}
Thus, $\|f_1(x)\|\leq c\|x\|$. So $f_1$ is bounded.

Now consider $f_2(x)=\alpha x(a)+\beta x(b)$ for some fixed $\alpha,\beta$. Let $r,s$ be scalars. Then we see that
\begin{align}
f_2(rx+sy)&=\alpha((rx+sy)(a))+\beta((rx+sy)(b))\\
&=\alpha((rx)(a)+(sy)(a))+\beta((rx)(b)+(sy)(b))\\
&=\alpha(rx)(a)+\alpha(sy)(a)+\beta(rx)(b)+\beta(sy)(b)\\
&=r\alpha x(a)+r\beta x(b)+s\alpha y(a)+s\beta y(b)\\
&=r(\alpha x(a)+\beta x(b))+s(\alpha y(a)+\beta y(b))\\
&=rf_2(x)+sf_2(y).
\end{align}
So $f_2$ is linear.

We now show that $f_2$ is bounded.
\begin{align}
\|f_2(x)\|=|\int_a^b\alpha x(a)+\beta
\end{align}
\end{exercise}

\end{document}