 \documentclass[a4paper]{article}
%% Language and font encodings
\usepackage[english]{babel}
\usepackage[utf8x]{inputenc}
\usepackage[T1]{fontenc}

%% Sets page size and margins
\usepackage[letterpaper, portrait, margin=1in,top=1in,bottom=1.5in]{geometry}

%% Useful packages
\usepackage{amsmath}
\usepackage{amssymb}
\usepackage{amsthm}
\usepackage{amsfonts}
\usepackage{mathrsfs}
\usepackage{tikz}
\usepackage{graphicx}
\usepackage[shortlabels]{enumitem}
\newenvironment{exercise}[1]{\textbf{#1.}}

\begin{document}

\begin{flushright}
Cory Glover\\
1/27/20\\
Math 540
\end{flushright}

\begin{center}
HW 12\\
2.9.1,4,5,9,10
\end{center}

\begin{exercise}{2.9.1}
We solve for the null space as follows:
\begin{align}
\begin{pmatrix}1&3&2\\-2&1&0\end{pmatrix}&\rightarrow\begin{pmatrix}1&3&2\\0&7&4\end{pmatrix}.
\end{align}
So $x_1+3x_2+2x_3=0$ and $7x_2+4x_3=0$. Thu,s $x_1+3x_2=-2x_3$. Substituting we get $7x_2-2x_1-6x_2=x_2-2x_1=0$. So $x_2=2x_1$. Substituting back into the first equation, we get that $x_1+3x_2+2x_3=x_1+2x_1+3x_3=3x_1+3x_3=0$. So $x_3=-x_1$. Thus the null space is spanned by $\begin{pmatrix}x_1&2x_1&-x_1\end{pmatrix}^T$ where $x_1\in\mathbb{R}$.
\end{exercise}

\begin{exercise}{2.9.4}
<<<<<<< HEAD
Let $\{f_1,f_2,f_3\}$ be the dual basis of $\{e_1,e_2,e_3\}$ for $\mathbb{R}^3$ where $e_1=(1,1,1), e_2=(1,1,-1),$ and $e_3=(1,-1,-1)$. Let $x=(1,0,0)$. So $x=e_1+e_3$. Thus, $f_1(x)=\frac{1}{2},f_2(x)=0,$ and $f_3(x)=\frac{1}{2}$. 
=======
Let $\{f_1,f_2,f_3\}$ be the dual basis of $\{e_1,e_2,e_3\}$ for $\mathbb{R}^3$ where $e_1=(1,1,1), e_2=(1,1,-1),$ and $e_3=(1,-1,-1)$. Let $x=(1,0,0)$. So $x=e_1+e_3$. Thus, $f_1(x)=f_1(e_1+e_3)=1, f_2(x)=f_2(e_1+e_3)=0$, and $f_3(x)=f_3(e_1+e_3)=1$. 
>>>>>>> ab5a990437ef93c7384eee894e3f536764cc5103
\end{exercise}

\begin{exercise}{2.9.5}
Let $f$ be a linear functional on an $n$-dimensional vector space $X$. Assume that $f$ is the zero function, then $\mathscr{N}(f)=X$. So the null space is $n$-dimensional.
<<<<<<< HEAD
Now assume that $f$ is not the zero function. We know that the $f$ maps to the real numbers (or complex numbers) which each have dimension one. Then by the rank-nullity theorem, we know that $n=\mathscr{N}(f)+1$. So the null space must have dimension $n-1$.
=======
Now assume that $f$ is not the zero function. Let $x\in N(f)$. Since $x\in X$, we can rewrite $x$ as $x=\sum_{i=1}^n\alpha_ix_i$. So, $f(x)=\sum_{i=1}^n\alpha_if(x_i)=0$. Thus, $\hat{\alpha}_1f(x_1)=\sum_{i=2}^n\alpha_if(x_i)$ where $\hat{\alpha}=-\alpha$. 
>>>>>>> ab5a990437ef93c7384eee894e3f536764cc5103
\end{exercise}

\begin{exercise}{2.9.9}
Let $X$ be the vector space of all real polynomials of a real variable and of degree less than a given $n$, together with $x=0$. Let $f(x)=x^{(k)}(a)$, the value of the $k^{th}$ derivative ($k$ fixed) of $x\in X$ at a fixed $a\in\mathbb{R}$. We show that this is a linear functional. It is clear that $f$ maps to the reals, so it suffices to show that it is linear. Let $x,y\in X$ and $\gamma,\lambda\in\mathbb{R}$. Then let $x=\alpha_0+\alpha_1 r+\dotsb+\alpha_nr^n$ and $y=\beta_0+\beta_1 r+\dotsb+\beta_n r^n$ where $\alpha_j$ and/or $\beta_j=0$ for all $j>l$ if $x$ (or $y$) has degree $l$. Then 
\begin{align}
f(\gamma x+\lambda y)&=f(\gamma(\alpha_0+\alpha_1 r+\dotsb+\alpha_nr^n)+\lambda(\beta_0+\beta_1 r+\dotsb+\beta_n r^n))\\
&=f((\gamma\alpha_0+\lambda\beta_0)+(\gamma\alpha_1+\lambda\beta_1)r+\dotsb+(\gamma\alpha_n+\lambda\beta_n)r^n)\\
&=((\gamma\alpha_1+\lambda\beta_1)+(\gamma\alpha_2+\lambda\beta_2)a+\dotsb+(\gamma\alpha_n+\lambda\beta_n)a^{n-1}\\
&=\gamma\alpha_1+\gamma\alpha_2a+\dotsb+\gamma\alpha_na^{n-1}+\lambda\beta_1+\lambda\beta_2a+\dotsb+\lambda\beta_na^{n-1}\\
&=\gamma(\alpha_1+\alpha_2a+\dotsb+\alpha_na^{n-1})+\lambda(\beta_1+\beta_2a+\dotsb+\beta_na^{n-1})\\
&=\gamma f(x)+\lambda f(y).
\end{align}
So $f$ is linear and thus a linear functional.
\end{exercise}

\begin{exercise}{2.9.10}
<<<<<<< HEAD
Let $Z$ be a proper subspace of an $n$-dimensional vector space $X$, and let $x_0\in X-Z$. We define a basis of $Z$ as $\{z_1,z_2,...,z_r\}$. Since $x_0\notin X-Z$, we extend the basis of $Z$ to $\{z_1,...,z_r,x_0\}$. Then we extend this basis to be a basis of $X$ by $\{z_1,...,z_r,x_0,x_1,...,x_k\}$. So every $x\in X$ can be written as $x=\sum_{i=1}^r\alpha_iz_i+\beta_0x_0+\sum_{i=1}^k\beta_ix_i$. Then we define the dual basis functions as $f_{v}(x)=\delta_{vx}$ as described in the chapter. From the chapter, we know that these are linear functionals. Then the basis functional, $f_{x_0}(x_0)=1$ but $f_{x_0}(z)=0$ for all $z\in Z$. 
=======
Let $Z$ be a proper subspace of an $n$-dimensional vector space $X$, and let $x_0\in X-Z$. Let $f$ be a functional defined by
\[f(x)=\begin{cases}1&x=x_0\\0&x\neq x_0\end{cases}.\]
We show that $f$ is linear. Let $\alpha,\beta\in \mathbb{R}$. We work by cases:
\begin{enumerate}
\item Let $x\in X$ and $x\neq x_0$. Then $f(x+x_0)=0$ and $f(x)+f(x_0)=1$. 
\end{enumerate}
>>>>>>> ab5a990437ef93c7384eee894e3f536764cc5103
\end{exercise}
\end{document}