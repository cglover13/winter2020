 \documentclass[a4paper]{article}
%% Language and font encodings
\usepackage[english]{babel}
\usepackage[utf8x]{inputenc}
\usepackage[T1]{fontenc}

%% Sets page size and margins
\usepackage[letterpaper, portrait, margin=1in,top=1in,bottom=1.5in]{geometry}

%% Useful packages
\usepackage{amsmath}
\usepackage{amssymb}
\usepackage{amsthm}
\usepackage{amsfonts}
\usepackage{mathrsfs}
\usepackage{tikz}
\usepackage{graphicx}
\usepackage[shortlabels]{enumitem}
\newenvironment{exercise}[1]{\textbf{#1.}}

\begin{document}

\begin{flushright}
Cory Glover\\
2/6/20\\
Math 540
\end{flushright}

\begin{center}
HW 13\\
2.10.2,4,6,10
\end{center}

\begin{exercise}{2.10.2}
Let $f$ and $g$ be bounded linear functionals with domains in a normed space $X$. Let $\alpha,\beta$ be nonzero scalars. We define a function $h=\alpha f+\beta g$. Since $h$ is defined in terms of $f$, $\mathscr{D}(h)\subset\mathscr{D}(f)$. Similarly, $\mathscr{D}(h)\subset\mathscr{D}(g)$. Thus, $\mathscr{D}(h)\subset\mathscr{D}(f)\cap\mathscr{D}(g)$.  Further, let $x\in\mathscr{D}(h)$. So $h(x)=\alpha f(x)+\beta g(x)$ is defined. So $x\in\mathscr{D}(f)\cap\mathscr{D}(g)$. Thus, $\mathscr{D}(h)=\mathscr{D}(f)\cap\mathscr{D}(g)$.

Since $h$ takes on values of $X$ and maps them to real-valued numbers (since it is a linear combination of real-valued numbers), $h$ is indeed a functional. It suffices to show it is bounded. Note that for all $x\in\mathscr{D}(h)$,
\begin{align}
|h(x)|&=|\alpha f(x)+\beta g(x)|\\
&\leq|\alpha f(x)|+|\beta g(x)|\\
&\leq|\alpha||f(x)|+|\beta||g(x)|\\
&\leq|\alpha|c\|x\|+|\beta|d\|x\|\\
&=(|\alpha|c+|\beta|d)\|x\|,
\end{align}
where $c,d\in\mathbb{R}$ since $f$ and $g$ are bounded linear functionals. Thus, $h$ is a bounded linear functional.
\end{exercise}

\begin{exercise}{2.10.4}
Let $X$ and $Y$ be normed spaces and $T_n\colon X\rightarrow Y$ ($n=1,2,\dotsb)$ bounded linear operators. Assume that $T_n\rightarrow R$. Let $\epsilon >0$ and choose $N$. Then when $n>N$, we see that $\|T_n-T\|\leq\epsilon$. Let $x\in \overline{B(x_0;\epsilon^*)}$. So $\|x-x_0\|\leq \epsilon^*$. Then we see that
\begin{align}
\|T_nx-Tx\|&=\|T_nx-Tx+Tnx_0-Tnx_0+Tx_0-Tx_0\|\\
&\leq\|T_nx-Tx-T_nx_0+Tx\|+\|T_nx_0-Tx_0\|\\
&=\|(T_n-T)(x-x_0)\|+\|(T_n-T)x_0\|\\
&\leq\|T_n-T\|\|x-x_0\|+\|T_n-T\|\|x_0\|\\
&<\epsilon(\epsilon^*+\|x_0\|).
\end{align}
We choose $\epsilon_{x_0}=\frac{\epsilon}{\epsilon^*+\|x_0\|}$ such that $\epsilon_{x_0}<\epsilon'$. Since $\epsilon$ is arbitrary, then $\|T_nx-Tx\|<\epsilon'$.
\end{exercise}

\begin{exercise}{2.10.6}
Let $X$ be the space of ordered $n$-tuples of real numbers and $\|x\|=\max_i|x_i|$ where $x=(x_1,...,x_n)$.  Thus the norm on the dual space is $\|f\|=\sup_{x\in X,x\neq 0}\frac{|f(x)|}{\max_i|x_i|}$.

%Now let $f$ be in the dual space of $X$. So $f$ is a bounded linear functional. Thus, 
%\[f(x)=f(\sum_{i=1}^n\alpha_ie_i)=\sum_{i=1}^n\alpha_if(e_i).\]
%Define $f(e_i)=\gamma_i$. 
%Note that
%\[|\gamma_i|=|f(e_i)|\leq\|f\|\|e_i\|=\|f\|\]
%since $f$ is bounded and since $\|e_i\|=1$. Thus we see that $\max_i|\gamma_i|\leq\|f\|$.
%
%Now note that \[|f(x)|=|f(\sum_{i=1}^n\alpha_ie_i)|=|\sum_{i=1}^n\alpha_if(e_i)|\leq\sum_{i=1}^n|\alpha_i||f(e_i)|\leq n\max_{i,j}|\alpha_i||\gamma_j|=\max_i\|n x\||\gamma_j|.\]
%Taking the maximum over all $x$ of norm $\frac{1}{n}$, we get that $\|f\|\leq\max_j|\gamma_j|$. So $\|f\|=\max_j|\gamma_j|$.
\end{exercise}

\begin{exercise}{2.10.10}
Let $X$ and $Y\neq\{0\}$ be normed vector spaces, with dim $X=\infty$. Let $B=(x_1,x_2,...)$ be a Hamel basis for $X$. Define a linear operator $T\colon X\rightarrow Y$. Let $y\in Y$ such that $y\neq 0$. Then we define $T$ as
\begin{align}
T(x_1)&=y\\
T(x_2)&=2y\\
&\vdots\\
T(x_i)&=iy\\
&\vdots
\end{align}
and $T(\alpha x_i)=\alpha iy$ for some scalar $\alpha$. Further, we define it such that $T(x)=T(\sum_{i=1}^\infty\alpha_ix_i)=\sum_{i=1}^\infty\alpha_iT(x_i)$. This function is clearly well-defined and linear.

Now we show that $T$ is not bounded. We know that $\|Tx\|=\|\sum_{i=1}^\infty\alpha_i(i)y\|\leq\sum_{i=1}^\infty\alpha_i i\|y\|=\infty$. So $T$ is not bounded.
\end{exercise}

\end{document}