 \documentclass[a4paper]{article}
%% Language and font encodings
\usepackage[english]{babel}
\usepackage[utf8x]{inputenc}
\usepackage[T1]{fontenc}

%% Sets page size and margins
\usepackage[letterpaper, portrait, margin=1in,top=1in,bottom=1.5in]{geometry}

%% Useful packages
\usepackage{amsmath}
\usepackage{amssymb}
\usepackage{amsthm}
\usepackage{amsfonts}
\usepackage{mathrsfs}
\usepackage{tikz}
\usepackage{graphicx}
\usepackage[shortlabels]{enumitem}
\newenvironment{exercise}[1]{\textbf{#1.}}

\begin{document}

\begin{flushright}
Cory Glover\\
Math 540\\
1/9/20
\end{flushright}

\begin{center}
\textbf{HW 1.1}\\
1.2.3,4,5,8
\end{center}

\begin{exercise}{1.2.3}
Let $x=(\xi_i)\in l^2$ and $y=(\eta_i)\in l^2$. 
Then the Cauchy-Schwarz inequality yields
\[\sum_{j=1}^\infty|\xi_j\eta_j|\leq\sqrt{\sum_{k=1}^\infty|\xi_k|^2}\sqrt{\sum_{m=1}^\infty|\eta_m|^2}.\]
Let $x=(\xi_1,...,\xi_n)$ and $y=(1,1,...,1)$ where $y$ has length $n$. Then the Cauchy Schwarz inequality gives
\begin{align}
\sum_{i=1}^n|\xi_i(1)|&\leq\sqrt{\sum_{k=1}^n|\xi_k|^2}\sqrt{\sum_{k=1}^n1^2}\\
\Bigl(\sum_{i=1}^n|\xi_i|\Bigr)^2&\leq n\sum_{k=1}^n|\xi_k|^2,
\end{align}
by squaring both sides and note that $\sum_{k=1}^n1^2=n$.
\end{exercise}

\begin{exercise}{1.2.4}
Consider the sequence $S_n=(\frac{1}{\log(n)})$. Since $\log(n)$ is increasing as $n\rightarrow\infty$, then $S_n$ converges to 0. However, note that $\sum_{n=1}^\infty\frac{1}{n}<\sum_{n=1}^\infty\frac{1}{\log(n)^p}$ for all $p>0$. Since $\sum_{n=1}^\infty\frac{1}{n}$ diverges, so does $S_n$ for all $p\geq 1$. So $S_n\notin l^p$ for all $p\geq1$.
\end{exercise}

\begin{exercise}{1.2.5}
Consider the sequence $S_n=(\frac{1}{n})$. It is commonly know that for $p=1$, then $\sum_{n=1}^\infty\frac{1}{n}$ diverges (i.e. the harmonic series). However we see that the series
$\sum_{n=1}^\infty2^n(\frac{1}{2^n})^p=\sum_{n=1}^\infty2^n(1-p)$. This series converges if and only if $p>1$. If this series converges, then by the Cauchy test, $\sum_{n=1}^\infty\frac{1}{n^p}$ converges. Thus for all $p>1$, $S_n\in l^p$. (This was found on accident on wikipedia while looking for convergence tests to check results. All work was done without referring to the result though.)
\end{exercise}

\begin{exercise}{1.2.8}
Let $A=\{0,1\}$ and $B=\{-1,0\}$. Then $D(A,B)=0$ but $A\neq B$ since $-1\notin A$ and $-1\in B$.
\end{exercise}
\end{document}