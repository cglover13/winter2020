 \documentclass[a4paper]{article}
%% Language and font encodings
\usepackage[english]{babel}
\usepackage[utf8x]{inputenc}
\usepackage[T1]{fontenc}

%% Sets page size and margins
\usepackage[letterpaper, portrait, margin=1in,top=1in,bottom=1.5in]{geometry}

%% Useful packages
\usepackage{amsmath}
\usepackage{amssymb}
\usepackage{amsthm}
\usepackage{amsfonts}
\usepackage{mathrsfs}
\usepackage{tikz}
\usepackage{graphicx}
\usepackage[shortlabels]{enumitem}
\newenvironment{exercise}[1]{\textbf{#1.}}

\begin{document}

\begin{flushright}
Cory Glover\\
Math 540\\
1/11/20
\end{flushright}

\begin{center}
\textbf{HW 3}\\
1.3.1,5,8,14,15
\end{center}

\begin{exercise}{1.3.1}
Let $x\in B(x_0;\epsilon)$. Then $d(x,x_0)<\epsilon$. Let $Y$ be a set such that for all $y\in Y$, $d(y,x_0)=\epsilon$. Let $y_0=\text{argmin}_{y\in Y} d(y,x)$. Define $\delta=d(y_0,x)$. Then there exists a ball $B(x;\delta/2)$ such that $x\in B(x;\delta/2)$. Since all the points $z\in B(x;\delta/2)$, and $d(x,y_0)>\delta/2$, then $z\in B(x_0;\epsilon)$. So there exists an open ball around every $x\in B(x_0;\epsilon)$. So an open ball is an open set.

 Consider the closed ball $\overline{B}(x_0;\epsilon)$. Let $x$ be an element of the complement of the closed ball. 
 Let $Y$ be a set such that for all $y\in Y$, $d(y,x_0)=\epsilon$. Let $y_0=\text{argmin}_{y\in Y} d(y,x)$. Define $\delta=d(y_0,x)$. Then there exists a ball $B(x;\delta/2)$ such that $x\in B(x;\delta/2)$. Since all the points $z\in B(x;\delta/2)$, and $d(x,y_0)>\delta/2$, then $z\notin \overline{B}(x_0;\epsilon)$.
 So the complement of the closed ball $\overline{B}(x_0;\epsilon)$ is open. So every closed ball is a closed set.
\end{exercise}

\begin{exercise}{1.3.5}
Consider the set $X$. From the properties of a topology (T1), $X$ is an open set. Note that $X^C=\emptyset$. By the properties of a topology (T1), $\emptyset$ is an open set. So $X$ is also a closed set.

Similarly, by the properties of a topology (T1), $\emptyset$ is an open set, but since $X$ is open and $\emptyset^C=X$, $\emptyset$ is also a closed set.

Consider a metric space $X=(X,d)$ where $d$ is the discrete metric. Let $Y\subset X$ and let $y\in Y$. Then $B(y;1)=\{y\}$ so $B(y;1)\in Y$ since $y\in Y$. So every subset of $X$ is open.

Now consider $Y^c$ and let $z\in Y^C$. Then $B(z;1)=\{z\}$. So $B(z;1)\in Y^C$. So $Y^C$ is open. Thus, $Y$ is closed.

So all subsets of $X$ are both open and closed.
\end{exercise}

\begin{exercise}{1.3.8}
Consider the space $X=([0,1]\cup\{2\},d)$ where $d$ is the Euclidean metrix. Then $\overline{B}(1,1)=[0,1]\cup\{2\}$ since $d(1,2)=1$. However, $2\notin\overline{B(1,1)}$. So the closure of a ball and a closed ball are not necessarily equal.
\end{exercise}

\begin{exercise}{1.3.14}
Assume that $T\colon X\rightarrow Y$ is continuous. Let $M\subset Y$ and $T^{-1}(M)$ be the inverse image of $M$. Let $M$ be closed. Then $M^C$ is open. Since $T$ is continuous, $T^{-1}(M^C)$ is also open by the continuous mapping theorem.. Let $m\in T^{-1}(M)^C$. Then $T(m)\in M^c$. So $T^{-1}(M)^C\subset T^{-1}(M^C)$. Now let $x\in T^{-1}(M^C)$. Then $Tx\in M^C$. So $Tx\notin M$. Thus, $x\notin T^{-1}(M)$. So $x\in T^{-1}(M)^c$. So $T^{-1}(M)^c=T^{-1}(M^c)$. Since $T^{-1}(M^C)$ is open, then $T^{-1}(M)^C$ is open. So $T^{-1}(M)$ is closed.

Let $T\colon X\rightarrow Y$ be a mapping from $X$ to $Y$. Let $M$ be a closed set in $Y$. Then $M^c$ is open. Assume that $T^{-1}(M)$ is also closed. Let $m\in M^C$. Since $M$ is closed, then $m$ has an open ball around it still contained in $M^C$. Then $T^{-1}(m)\notin T^{-1}(M).$ So $T^{-1}(m)\in T^{-1}(M)^c$.  Thus, $T^{-1}(m)$ has open ball around it still contained in $T^{-1}(M)^c$ since $T^{-1}(M)$ is closed. Since $m$ is arbitrary, $T^{-1}(M)^c$ is open. So the inverse image of every open set in $Y$ is an open set in $X$. Thus, $T$ is continuous.
\end{exercise}

\begin{exercise}{1.3.15}
Let $X=(\mathbb{R},d)$ where $d$ is the discrete metric and $Y=(\mathbb{R},\overline{d})$ where $\overline{d}$ is the Euclidean metric. Define $T(A)=A$ where $A$ is a subset of the real line. Then for every open set $A\subset Y$, $T^{-1}(Y)$ is open (since every subset in $X$ is open). However, consider the subset $[0,1]\in X$. This set is open, however $T(X)$ is not open.
\end{exercise}
\end{document}