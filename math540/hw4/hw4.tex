 \documentclass[a4paper]{article}
%% Language and font encodings
\usepackage[english]{babel}
\usepackage[utf8x]{inputenc}
\usepackage[T1]{fontenc}

%% Sets page size and margins
\usepackage[letterpaper, portrait, margin=1in,top=1in,bottom=1.5in]{geometry}

%% Useful packages
\usepackage{amsmath}
\usepackage{amssymb}
\usepackage{amsthm}
\usepackage{amsfonts}
\usepackage{mathrsfs}
\usepackage{tikz}
\usepackage{graphicx}
\usepackage[shortlabels]{enumitem}
\newenvironment{exercise}[1]{\textbf{#1.}}

\begin{document}

\begin{flushright}
Cory Glover\\
Math 540\\
1/14/19
\end{flushright}

\begin{center}
HW 1.4.2,4,5
\end{center}

\begin{exercise}{1.4.2}
Let $(x_n)$ be a Cauchy sequence with a convergent subsequence $x_{n_k}\rightarrow x$. So $\epsilon>0$, there exists a $K$ such that for $k>K$, $d(x_{n_k},x)<\frac{\epsilon}{2}$. Further, since $(x_n)$ is Cauchy, we know that for some $N$, there exists $n,n_k>N$ such that $d(x_{n_k},x_n)<\frac{\epsilon}{2}$. So for $\max(N,K)$, we see by the triangle inequality that
\[d(x_n,x)\leq d(x_n,x_{n_k})+d(x_{n_k},x)<\frac{\epsilon}{2}+\frac{\epsilon}{2}=\epsilon.\]
So $x_n\rightarrow x$.
\end{exercise}

\begin{exercise}{1.4.4}
Let $(x_n)$ be a Cauchy sequence. Thus, for every $\epsilon>0$, there exists $N$ such that for $m,n>N$, $d(x_n,x_m)<\epsilon$. Let $M$ be an element of the metric space $X$ such that $d(M,x_n)>1$.  
\end{exercise}

\begin{exercise}{1.4.5}
Boundedness is not necessary for a sequence to be either Cauchy or convergent. Consider the sequence on the real line $(0,1,0,1,0,1,0,1,...)$. This sequence is bounded by $1$ but it does not converge to any number and the distance between elements does not approach $\epsilon>0$. 
\end{exercise}
\end{document}