\documentclass[a4paper]{article}
%% Language and font encodings
\usepackage[english]{babel}
\usepackage[utf8x]{inputenc}
\usepackage[T1]{fontenc}

%% Sets page size and margins
\usepackage[letterpaper, portrait, margin=1in,top=1in,bottom=1.5in]{geometry}

%% Useful packages
\usepackage{amsmath}
\usepackage{amssymb}
\usepackage{amsthm}
\usepackage{amsfonts}
\usepackage{mathrsfs}
\usepackage{tikz}
\usepackage{graphicx}
\usepackage[shortlabels]{enumitem}
\newenvironment{exercise}[1]{\textbf{#1.}}

\begin{document}

\begin{flushright}
Cory Glover\\
1/16/20\\
Math 540
\end{flushright}

\begin{center}
HW 8\\
2.3.3,8,10\\
2.4.2,4,6
\end{center}

\begin{exercise}{2.3.3}
Let $X$ be $l^\infty$ and let $Y$ be the subset of all sequences with only finitely many nonzero terms. We first show that $Y$ is a subspace of $l^\infty$. We first note that the zero sequence has no nonzero terms, and thus is in $Y$. So $Y$ is nonempty.  Let $x=(x_i)$ and $y=(y_i)$ be elements of $Y$. Let $x_n$ be the last nonzero term of $x$ and let $y_m$ be the last nonzero term of $y$. Then $\alpha x+\beta y$ has a last nonzero term at either $n$ or $m$ ($\alpha x_n$ or $\beta y_m$ respectively). Thus, $x+y\in Y$. So $Y$ is a subspace.

However consider the sequence $x_n=(1+\frac{1}{n},1+\frac{1}{n},...,1+\frac{1}{n},0,...)$ where the first $n$ entries of each tuple are $1+\frac{1}{n}$ and the rest are zero. Then for $\epsilon>0$, there exists an $N$ such that when $n>N$, $\|x_n-(1,1,1,...)\|\ leq\epsilon$. Since $(1,1,1,...)\notin Y$, we see that $Y$ is not closed, and thus is not a closed subspace.
\end{exercise}

\begin{exercise}{2.3.8}
Let $X$ be a normed space where absolute convergence implies convergence. Let $(x_n)\in X$ be a Cauchy sequence. Then for $\epsilon^*>0$, there exists an $N$ such that when $m,n>N$, then $\|x_n-x_m\|<\epsilon^*$. We then define $y_r=x_r-x_{r-1}$. Since $(x_n)$ is Cauchy, for all $r>N+1$, $\|y_n\|<\epsilon^*$. We choose a subsequence $(y_{r_k})$ such that $\|y_{r_k}\|<\frac{1}{k^2}$. We can do this by redefining $\epsilon$ at each term $y_r$ and only choosing elements such that $\|y_{r_k}\|<\frac{1}{k^2}$ for our sequence (this will be sequential since the distance between elements is shrinking). Then we know that $\sum_{i=1}^\infty \|y_{r_i}\|<\sum_{i=1}^\infty\frac{1}{k^2}$. Since $\sum_{i=1}^\infty\frac{1}{k^2}$ converges, we know that $\sum_{i=1}^\infty \|y_{r_i}\|$ converges. Then since every absolutely convergent sequence converges in $X$, we know that $\sum_{i=1}^\infty y_{r_k}$ converges to some $s$. Note that $\sum_{i=1}^\infty y_{r_i}$ is a telescoping sum, such that $\sum_{i=1}^\infty y_{r_i}=\lim_{k\rightarrow\infty} x_{r_k}-x_{r_1}$. Thus, $\lim_{k\rightarrow\infty} x_{r_k}-x_{r_1}=s$. Thus, $\lim_{k\rightarrow\infty} x_{r_k}=s+x_{r_1}$. We note that $x_{r_1}\in X$ since $x_{r_1}\in (x_r)$. Further, we know that $s\in X$ since $\sum_{i=1}^\infty y_{r_i}$ converges to $s$. Since $X$ is a vector space, $s+x_{r_1}\in X$. Thus, $(x_{r_k})$ converges. Since $(x_r)$ is Cauchy, and it has a convergent subsequence, $(x_r)$ must converge. Thus, $X$ is complete.
\end{exercise}

\begin{exercise}{2.3.10}
Assume that $X$ is a normed space with a Schauder basis. Thus, for all $x\in X$, $x=\sum_{k=1}^\infty \alpha_ke_k$ where $(e_n)$ is the Schauder basis for $X$. Consider the set $X'\subset X$ where each $x'$ can be represented as $\sum_{k=1}^\infty\beta_ke_k$ where $\beta_k\in\mathbb{Q}$. Let $x\in X$ such that $x=\sum_{i=1}^\infty \alpha_ix_i$. Recall that $\mathbb{Q}$ is dense in $\mathbb{R}$. Choose $x'=\sum_{i=1}^\infty \beta_ie_i\in X'$ such that for every $\alpha_i$ and $\beta_i$, $\|\alpha_i-\beta_i\|<\frac{\epsilon_i}{\|\sum_{i=1}^\infty e_i\|}$ for every $\epsilon_i>0$. Let $\epsilon=\max_i\frac{\epsilon_i}{\|\sum_{i=1}^\infty e_i\|}$. Then,
\[\|x'-x\|=\|\sum_{i=1}^\infty\beta_ie_i-\sum_{i=1}^\infty\alpha_ie_i\|=\|\sum_{i=1}^\infty(\beta_i-\alpha_i)e_i\|\leq\|\sum_{i=1}^\infty\|\beta_i-\alpha_i\|e_i\|<\|\epsilon\sum_{i=1}^\infty e_i\|=\epsilon\|\sum_{i=1}^\infty e_i\|=\max_i\epsilon_i.\]
So $x'\in B(x;\max_i\epsilon_i)$. Thus, $x\in\overline{X'}$. Since $x$ was arbitrary, $\overline{X'}=X$. Further, since $X'$ is countable, $X$ is separable.
\end{exercise}

\begin{exercise}{2.4.2}
Let $X=\mathbb{R}^2$ and $x_1=(1,0)$ and $x_2=(0,1)$. Note that for $\alpha_1$ and $\alpha_2$,
\[\|\alpha_1x_1+\alpha_2x_2\|=\|\alpha_1(1,0)+\alpha_2(0,1)\|=\|(\alpha_1,\alpha_2)\|=\sqrt{\alpha_1^2+\alpha_2^2}.\]
Note that the largest $c$ in equation (1) will bring equality. So we solve for $c$ as follows:
\begin{align}
\sqrt{\alpha_1^2+\alpha_2^2}&=c|\alpha_1|+c|\alpha_2|\\
c&=\frac{\sqrt{\alpha_1^2+\alpha_2^2}}{|\alpha_1|+|\alpha_2|}.
\end{align}
So the largest $c$ could be is $c=\frac{\sqrt{\alpha_1^2+\alpha_2^2}}{|\alpha_1|+|\alpha_2|}$.

Now let $X=\mathbb{R}^3$ and $x_1=(1,0,0), x_2=(0,1,0)$ and $x_3=(0,0,1)$. Following the same procedure as above, we solve for $c$.
\begin{align}
\|\alpha_1(1,0,0)+\alpha_2(0,1,0)+\alpha_3(0,0,1)\|&=\|(\alpha_1,\alpha_2,\alpha_3)\|\\
&=\Bigl(\alpha_1^3+\alpha_2^3+\alpha_3^3\Bigr)^{1/3}\\
&=c(|\alpha_1|+|\alpha_2|+|\alpha_3|).
\end{align}
So the largest $c$ could be is $c=\frac{\Bigl(\alpha_1^3+\alpha_2^3+\alpha_3^3\Bigr)^{1/3}}{|\alpha_1|+|\alpha_2|+|\alpha_3|}$.
\end{exercise}

\begin{exercise}{2.4.4}
Let $X$ be a vector space and let $\|\cdot\|$ and $\|\cdot\|_*$ be equivalent norms on $X$.  This means that there exists positive $a$ and $b$ such that $a\|x\|_*\leq\|x\|\leq b\|x\|_*$ for all $x\in X$. Define $\mathscr{T}_1$ as the open subsets of $X$ with respect to $\|\cdot\|$ and $\mathscr{T}_2$ the open subsets of $X$ with respect to $\|\cdot\|_*$.. Now let $X_2\in \mathscr{T}_2$ be an open subset with respect to $\|\cdot\|_*$.  We want to show that $X_2$ is an open subset with respect to $\|\cdot\|$. Let $x_0\in B(x;a\epsilon)$ where this is a ball with respect to $\|\cdot\|$, $x\in X_2$ and $\epsilon>0$.  Then
\[a\|x-x_0\|_*\leq\|x-x_0\|<a\epsilon.\] 
So $x_0\in B^*(x;\epsilon)$. Thus, $B(x;a\epsilon)\subseteq B^*(x;\epsilon)$. So every open ball with respect to $\|\cdot\|$ around $x\in X_2$ is in $X_2$ since it is a subset of a ball with respect to $\|\cdot\|_*$. So $X_2$ is an open subset with respect to $\|\cdot\|$. Thus, $X_2\in\mathscr{T}_1$. So $\mathscr{T}_2\subseteq\mathscr{T}_1$.

Now let $X_1\in\mathscr{T}_1$ be an open subset with respect to $\|\cdot\|$. Recall that since $\|\cdot\|$ and $\|\cdot\|_*$ are equivalent, there exists $a'$ and $b'$ that are positive such that $a'\|x\|\leq\|x\|_*\leq b'\|x\|$ for all $x\in X$. Let $x\in X_1$, $\epsilon>0$ and $x_0\in B^*(x;a\epsilon)$ where $B^*$ is a ball with respect to $\|\cdot\|_*$. Then we see that
\[a'\|x-x_0\|\leq\|x-x_0\|_*<a\epsilon.\]
So $x_0\in B(x;\epsilon)$ where $B$ is a ball with respect to $\|\cdot\|$. So $B^*(x;a\epsilon)\subseteq B(x;\epsilon)$. Thus every open ball with respect to $\|\cdot\|_*$ around $x\in X_1$ is in $X_1$. So $X_1$ is an open subset with respect to $\|\cdot\|_*$. So $X_1\in\mathscr{T}_2$. Thus, $\mathscr{T}_1\subseteq\mathscr{T}_2$.

Hence, $\mathscr{T}_1=\mathscr{T}_2$.
\end{exercise} 

\begin{exercise}{2.4.6}
Let $X$ be the vector space of ordered $n$-tuples of numbers. Let $x=(x_1,...,x_n)\in X$. Then $\|x\|_\infty^2=\max_i|x_i|^2\leq\sum_{i=1}^n|x_i|^p=\|x\|_2^2$. Thus,
\begin{align}
\|x\|_\infty^2&\leq\|x\|_2^2\\
\|x\|_\infty\|&\leq\|x\|_2.
\end{align}
Further, $n^2\max_i|x_i|^2\geq n\max_i|x_i|^2\geq\sum_{i=1}^n|x_i|^2$. Thus,
\begin{align}
n^2\|x\|^2_\infty&\geq\|x\|_2^2\\
n\|x\|_\infty&\geq\|x\|_2.
\end{align}
So $\|x\|_\infty\leq\|x\|_2\leq n\|x\|_\infty$. Since $1,n>0$, the $\|\cdot\|_\infty$ and $\|\cdot\|_2$ are equivalent by definition 2.4-4.
\end{exercise}
\end{document}