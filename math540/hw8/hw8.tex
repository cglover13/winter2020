\documentclass[a4paper]{article}
%% Language and font encodings
\usepackage[english]{babel}
\usepackage[utf8x]{inputenc}
\usepackage[T1]{fontenc}

%% Sets page size and margins
\usepackage[letterpaper, portrait, margin=1in,top=1in,bottom=1.5in]{geometry}

%% Useful packages
\usepackage{amsmath}
\usepackage{amssymb}
\usepackage{amsthm}
\usepackage{amsfonts}
\usepackage{mathrsfs}
\usepackage{tikz}
\usepackage{graphicx}
\usepackage[shortlabels]{enumitem}
\newenvironment{exercise}[1]{\textbf{#1.}}

\begin{document}

\begin{flushright}
Cory Glover\\
1/16/20\\
Math 540
\end{flushright}

\begin{center}
HW 8\\
2.3.3,8,10\\
2.4.2,4,6
\end{center}

\begin{exercise}{2.3.3}
Let $X$ be $l^\infty$ and let $Y$ be the subset of all sequences with only finitely many nonzero terms. We first show that $Y$ is a subspace of $l^\infty$. We first note that the zero sequence has no nonzero terms, and thus is in $Y$. So $Y$ is nonempty.  Let $x=(x_i)$ and $y=(y_i)$ be elements of $Y$. Let $x_n$ be the last nonzero term of $x$ and let $y_m$ be the last nonzero term of $y$. Then $\alpha x+\beta y$ has a last nonzero term at either $n$ or $m$ ($\alpha x_n$ or $\beta y_m$ respectively). Thus, $x+y\in Y$. So $Y$ is a subspace.

However consider the sequence $x_n=(1,1,0,...)$ where the first $n$ entries of each tuple is 1 and the rest are zero. Then for $\epsilon>0$, there exists an $N$ such that when $n>N$, $\|x_n-(1,1,1,...)\|\leq\epsilon$. Since $(1,1,1,...)\notin Y$, we see that $Y$ is not closed, and thus is not a closed subspace.
\end{exercise}

\begin{exercise}{2.3.8}
Let $X$ be a normed space where absolute convergence implies convergence. Let $(x_n)\in X$ be a Cauchy sequence. Then for $\epsilon>0$, there exists an $N$ such that when $m,n>N$, then $\|x_n-x_m\|<\epsilon$. So for some $x\in X$ and $\epsilon'>0$, then there exists an $N$ such that when $n>N$,
\begin{align}
\|x_n-x\|&=\|x_n-x_m+x_m-x\|\\
&\leq\|x_n-x_m\|+\|x_m-x\|
\end{align}
\end{exercise}

\end{document}