\documentclass[a4paper]{article}
%% Language and font encodings
\usepackage[english]{babel}
\usepackage[utf8x]{inputenc}
\usepackage[T1]{fontenc}

%% Sets page size and margins
\usepackage[letterpaper, portrait, margin=1in,top=1in,bottom=1.5in]{geometry}

%% Useful packages
\usepackage{amsmath}
\usepackage{amssymb}
\usepackage{amsthm}
\usepackage{amsfonts}
\usepackage{mathrsfs}
\usepackage{tikz}
\usepackage{graphicx}
\usepackage[shortlabels]{enumitem}
\newenvironment{exercise}[1]{\textbf{#1.}}

\begin{document}

\begin{flushright}
Cory Glover\\
1/24/20\\
Math 540
\end{flushright}

\begin{center}
HW 9//
2.5.2,9.10
\end{center}

\begin{exercise}{2.5.2}
Let $X$ be a discrete metric space with infinitely many points. Consider the sequence $(x_1,x_2,x_3,...)\in X$. Note that for any $x_i$ and $x_{i+1}$, then $d(x_i,x_{i+1})=1$. Thus, we choose an arbitrary subsequence $(x_i,x_{i+1},x_{i+2},...)$ which may or may not be infinite. Then for $\epsilon=\frac{1}{2}$, then for every $k$ with $x_k$ an element of the subsequence, $d(x_k,x_{k+1})=1>\epsilon$. So the subsequence does not converge. Since the subsequence was arbitrary, $X$ is not compact.
\end{exercise}

\begin{exercise}{2.5.9}
Let $X$ be a compact metric space and that $M\subset X$ is closed. Let $(x_n)\in M$. Since $X$ is compact, we know that $(x_n)$ has a subsequence $(x_{n_k})$ which converges to some $x$. Since $(x_{n_k})$ is a subsequence of $(x_n)$, each element of $(x_{n_k})$ is in $M$. Further, since $M$ is closed, $x\in M$. Thus, since $(x_n)$ was arbitrary, $M$ must be compact.
\end{exercise}

\begin{exercise}{2.5.10}
Let $X$ and $Y$ be metric spaces. Let $X$ be compact and let $T\colon X\rightarrow Y$ be bijective and continous. We want to show that $T$ is a homeomorphism (that is $T$ has a continuous inverse). Consider the function $T^{-1}$. Let $x,x_0\in X$ such that for every $\epsilon>0$, $d(x,x_0)<\epsilon$. Since $X$ is compact, there exists some sequence $(x_n)$ where $x,x_0\in(x_n)$ which has a convergent subsequence (namely $(x,x_0)$). We write $x=T^{-1}(y)$ and $x_0=T^{-1}(y_0)$ for some $y,y_0\in Y$, since $T$ is bijective. Since $T$ is continuous, $T(T^{-1}(y),T^{-1}(y_0))$ maps to some subsequence $(y_n,y_0)$ which converges. So $d(y_n,y_0)<\delta$ for some $\delta$. Thus, $T^{-1}$ is continuous. 
\end{exercise}

\end{document}